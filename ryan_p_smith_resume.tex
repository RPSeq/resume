%-------------------------
% Resume in Latex
% Author : Sourabh Bajaj
% Modified by: Ryan P Smith
% License : MIT
%------------------------

\documentclass[letterpaper,hidelinks,11pt]{article}

\usepackage{latexsym}
\usepackage[empty]{fullpage}
\usepackage{titlesec}
\usepackage{marvosym}
\usepackage[usenames,dvipsnames]{color}
\usepackage{verbatim}
\usepackage{enumitem}
\usepackage[pdftex]{hyperref}
\usepackage{fancyhdr}

\pagestyle{fancy}
\fancyhf{} % clear all header and footer fields
\fancyfoot{}
\renewcommand{\headrulewidth}{0pt}
\renewcommand{\footrulewidth}{0pt}

% Adjust margins
\addtolength{\oddsidemargin}{-0.375in}
\addtolength{\evensidemargin}{-0.375in}
\addtolength{\textwidth}{1in}
\addtolength{\topmargin}{-.5in}
\addtolength{\textheight}{1.0in}

\urlstyle{same}

\raggedbottom
\raggedright
\setlength{\tabcolsep}{0in}

% Sections formatting
\titleformat{\section}{
  \vspace{-4pt}\scshape\raggedright\large
}{}{0em}{}[\color{black}\titlerule \vspace{-5pt}]

%-------------------------
% Custom commands
\newcommand{\resumeItem}[2]{
  \item\small{
    \textbf{#1}{: #2 \vspace{-2pt}}
  }
}

\newcommand{\publicationItem}[2]{
  \item\small{
    \textbf{#1}{#2 \vspace{-2pt}}
  }
}

\newcommand{\resumeSubheading}[4]{
  \vspace{-1pt}\item
    \begin{tabular*}{0.97\textwidth}{l@{\extracolsep{\fill}}r}
      \textbf{#1} & #2 \\
      \textit{\small#3} & \textit{\small #4} \\
    \end{tabular*}\vspace{-5pt}
}

\newcommand{\resumeSubItem}[2]{\resumeItem{#1}{#2}\vspace{-4pt}}

\renewcommand{\labelitemii}{$\circ$}

\newcommand{\resumeSubHeadingListStart}{\begin{itemize}[leftmargin=*]}
\newcommand{\resumeSubHeadingListEnd}{\end{itemize}}
\newcommand{\resumeItemListStart}{\begin{itemize}}
\newcommand{\resumeItemListEnd}{\end{itemize}\vspace{-5pt}}

%-------------------------------------------
%%%%%%  CV STARTS HERE  %%%%%%%%%%%%%%%%%%%%%%%%%%%%

\begin{document}

%----------HEADING-----------------
\begin{tabular*}{\textwidth}{l@{\extracolsep{\fill}}r}
  \textbf{{\Large Ryan P Smith}} \\   
  \href{https://github.com/RPSeq}{github.com/RPSeq} & \href{mailto:ryan.smith.p@gmail.com}{ryan.smith.p@gmail.com}\\
  \href{http://www.linkedin.com/in/rpseq}{linkedin.com/in/rpseq} & +1-319-899-0190 \\
\end{tabular*}

%-----------EDUCATION-----------------
\section{Education}
  \resumeSubHeadingListStart
  
    \resumeSubheading
      {Washington University}{St Louis, MO}
      {PhD (In Progress), Molecular Genetics and Genomics}{Aug. 2014 -- Present}
      
    \resumeSubheading
      {University of Iowa}{Iowa City, IA}
      {BS, Microbiology and Informatics;  GPA: 3.8}{Aug. 2009 -- May. 2014}
      
  \resumeSubHeadingListEnd

%-----------EXPERIENCE-----------------
\section{Experience}
  \resumeSubHeadingListStart
  
    \resumeSubheading
      {The McDonnell Genome Institute, Washington University}{St Louis, MO}
      {Graduate Research Scientist, Ira Hall Lab, Computational Genetics}{Feb 2015 - Present}
      
      \resumeItemListStart
      
        \resumeItem{Distributed Computing, Bioinformatics}
          {Leveraging a high-performance distributed computing cluster to process population-scale genome
           sequencing data, improving our understanding of the causes and consequences of structural
           variation in human genomes. Most work is done in a Linux environment with custom Bash pipelines, often using bioinformatics tools combining Bayesian statistics and machine learning approaches.}
       
        \resumeItem{Single-cell Sequencing}
          {Developing novel computational and molecular biology methods to sequence the genomes of single
           mammalian neurons in collaboration with The Scripps Research Institute at University of
           California, San Diego. (Processing up to 10 TB of raw Illumina sequencing data within a 36 hour period.)}
       
        \resumeItem{Teaching Assistant}
          {Advised students in characterizing the genome sequences of novel bacteriophages. Hands-on computer lab course using a series of bioinformatics tools to identify genes and predict their function, as well as classifying unknown phages into existing phylogenetic groups.}
          
      \resumeItemListEnd
      
    \resumeSubheading
      {University of Iowa}{Iowa City, IA}
      {Undergraduate Research Fellow, Adam Dupuy Lab, Cancer Genetics}{Aug 2010 - May 2014}
      
      \resumeItemListStart
      
        \resumeItem{Genome Editing}
          {Designed methods for viral genetic engineering in mouse models of human cancers and a
           sequencing method for detecting resulting transgene insertions.}
       
        \resumeItem{Bioinformatics}
          {Processed high-throughput genome sequencing data using Linux bioinformatics tools followed
           by ad-hoc statistical analyses and data visualization in Python and R.}
       
      \resumeItemListEnd
      
  \resumeSubHeadingListEnd

%--------INTERESTS------------
\section{Relevant Interests}

{Data science, distributed computing, data visualization, bioinformatics, genetics. I enjoy learning new technologies and tinkering with complex systems. After several years of genetics and bioinformatics training in academia, I am currently seeking industry experience with specific interests in data science and software engineering.}

%--------PROGRAMMING SKILLS------------
\section{Programming Skills}
  \resumeSubHeadingListStart
  
        \resumeItem{Languages}
          {Bash, Python, R (ggplot), awk, SQL, C++, JavaScript, HTML, PHP, \LaTeX}
          
        \resumeItem{Technologies}
          {Linux/Unix, OSX, sed, git, Docker, high-performance distributed computing (IBM Platform LSF, Oracle Grid Engine)}
          
  \resumeSubHeadingListEnd

%--------Publications------------
\section{Publications}
  \resumeSubHeadingListStart
  
    \publicationItem{Jesse D. Riordan, Luke Drury, Ryan P. Smith*, Ben T. Brett, Adam J. Dupuy}
    { ``Sequencing Methods and Datasets to Improve Functional Interpretation of Sleeping Beauty Mutagenesis Screens'' BMC Genomics, 15(1): 1150. December 2014}
    
    \publicationItem{Ryan P. Smith*, Jesse D. Riordan, Charlotte R. Feddersen, Adam J. Dupuy}
    { ``A Hybrid Adenoviral Vector System Achieves Efficient Long-term Gene Expression in the Liver via PiggyBac Transposition'' Human Gene Therapy, 26(6):377-85. June 2015}
    
    \publicationItem{Jennifer L. Hazen, Michael A. Duran, Ryan P. Smith*, Alberto R. Rodriguez, Greg S. Martin, Sergey Kupriyanov, Ira M. Hall, Kristin K. Baldwin}
    { ``Using Cloning to Amplify Neuronal Genomes for Whole-Genome Sequencing and Comprehensive Mutation Detection and Validation'' Genomic Mosaicism in Neurons and Other Cell Types. Neuromethods, vol 131. September 2017}
    
  \resumeSubHeadingListEnd

%-------------------------------------------
\end{document}
